\documentclass[11pt]{article}
\usepackage[utf8]{inputenc}
\usepackage[margin=1.0in]{geometry}

\title{CIS 3715 Final Project Proposal}
\author{Leomar Durán}
\date{March 2022}

\usepackage{setspace}
\singlespacing

\usepackage{hyperref}

\usepackage{tabularx}
\usepackage{booktabs}
\newcommand*\ra[1]{\renewcommand*\arraystretch{#1}}%
\ra{1.27273}

% flushes heading numbers into the margin
\usepackage{titlesec}
\titlelabel{\llap{\thetitle\quad}}

% for loading the bibliography
\usepackage[
    style=ieee
]{biblatex}
\addbibresource{main.bib}
\defbibheading{bibliography}[\refname]{%
  \section{#1}%
}%

% SI units, added pixels
\usepackage[group-separator={,},per-mode=symbol]{siunitx}
\DeclareSIUnit\pixel{px}

% math delimiters
\usepackage{mathtools}%
\DeclarePairedDelimiter\brao()%

% macros for delimiters in text
\newcommand*\DeclareTextPairedDelimiters[3]{%
    \newcommand*#1[1]{#2##1#3}%
}
\DeclareTextPairedDelimiters\tbrao()%
\DeclareTextPairedDelimiters\tangle{\(<\)}{\(>\)}%

\begin{document}

\section{Project title and student names}
\begin{itemize}
    \item
        \textbf{Project title:}
        Using satellite imagery to train a model for identifying the type of landmarks.
    \item
        \textbf{Student names} (1)
        \begin{itemize}
            \item
                Leomar Durán
        \end{itemize}
\end{itemize}

\section{Revision History}

\begin{tabularx}\linewidth{@{}rlrX@{}}
    \toprule
        Revision \#
            & Author
            & Revision date
            & Comments
    \\*
    \midrule
        1.6.0
            & Leomar Durán
            & 2022-03-25t22:16
            & added revision history
    \\*
        1.5.0
            & Leomar Durán
            & 2022-03-25t21:47
            & added more about the dataset, grammatical fixes
    \\*
        1.4.0
            & Leomar Durán
            & 2022-03-25t02:45
            & started bibliograph, problem statement
    \\*
        1.3.0
            & Leomar Durán
            & 2022-03-25t02:10
            & motivation, flushed heading numbers into margin
    \\*
        1.2.0
            & Leomar Durán
            & 2022-03-25t02:05
            & sections, title, students
    \\*
        1.1.0
            & Leomar Durán
            & 2022-03-25t01:55
            & page specifications
    \\*
        1.0.0
            & Leomar Durán
            & 2022-03-25t01:51
            & starting proposal
    \\*
    \bottomrule
\end{tabularx}

\section{Introduction section}

\subsection{Motivation}

The sciences of geomatics and land surveying interest me as hobbies.
I really enjoy the idea of collecting data about the terrain,
whether it be rural or urban,
and working with that data to find solutions to problems
or even just for fun.

\subsection{Problem}

We are given
The SAT-4 airborne dataset%
\cite{Basu2015a}.
This data is hosted by the Louisiana State University's Division of Computer Science and Engineering%
\footnote{%
    \tangle{\url{http://csc.lsu.edu/~saikat/deepsat/}}%
}
and can be downloaded directly from the Google Drive%
\footnote{%
    \tangle{\url{https://drive.google.com/u/0/uc?export=download&confirm=sWVM&id=0B0Fef71_vt3PUkZ4YVZ5WWNvZWs}}%
}
along with the SAT-6 airborne dataset.

The dataset consists of \(\num{400000}\) example tiles
taken from satellite imagery originally from the National Agriculture Imagery Program \tbrao{NAIP} dataset.
Each sample has features representing the pixels of a \(\SI[parse-numbers=false]{\brao{28\times28}\!}\pixel\) image 
multiplied by the channels for red, green, blue and near infrared \tbrao{NIR}.
According to \textcite{Basu2015a},
these tiles represent ``different landscapes like rural areas, urban areas, densely forested, mountainous terrain, small to large water bodies",
so these as a disjoint set of landscapes would make for appropriate labels.

\subsection{Related works}

\section{Proposed work section}
\section{Timeline}

\printbibliography

\end{document}
