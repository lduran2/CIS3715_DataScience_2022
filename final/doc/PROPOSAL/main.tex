\documentclass[11pt]{article}
\usepackage[utf8]{inputenc}
\usepackage[margin=1.0in]{geometry}

\title{CIS 3715 Final Project Proposal}
\author{Leomar Durán}
\date{March 2022}

% line spacing, paragraph indentation
\usepackage{setspace}
\singlespacing
\setlength\parindent{0.25in}

% hyperlinks
\usepackage{hyperref}

% remove level 1 margin on any list
\usepackage[inline]{enumitem}
\setlist[1]{leftmargin=0in}

% table width, rules, row spacing
\usepackage{tabularx}
\usepackage{booktabs}
\newcommand*\ra[1]{\renewcommand*\arraystretch{#1}}%
\ra{1.27273}

% flushes heading numbers into the margin
\usepackage{titlesec}
\titlelabel{\llap{\thetitle\quad}}

% for loading the bibliography
\usepackage[%
    style=ieee%
]{biblatex}
\addbibresource{main.bib}
\defbibheading{bibliography}[\refname]{%
  \section{#1}%
}%

% SI units, added pixels
\usepackage[%
    group-separator={,},%
    per-mode=symbol%
]{siunitx}
\DeclareSIUnit\pixel{px}
\DeclareSIUnit\example{examples}
\DeclareSIUnit\channel{channels}
\DeclareSIUnit\byte{byte}

% math delimiters
\usepackage{mathtools}%
\DeclarePairedDelimiter\brao()%

% macros for delimiters in text
\newcommand*\DeclareTextPairedDelimiters[3]{%
    \newcommand*#1[1]{#2##1#3}%
}
\DeclareTextPairedDelimiters\tbrao()%
\DeclareTextPairedDelimiters\tbrac[]%
\DeclareTextPairedDelimiters\tangle{\(<\)}{\(>\)}%

\begin{document}

\section*{Project title and student names}
\begin{itemize}
    \item
        \textbf{Project title:}
        Using satellite imagery to train a model for identifying the type of landmarks.
    \item
        \textbf{Student names} (1)
        \begin{itemize}
            \item
                Leomar Durán
        \end{itemize}
\end{itemize}

\section*{Revision History}

\begin{tabularx}\linewidth{@{}rlrX@{}}
    \toprule
        Revision \#
            & Author
            & Revision date
            & Comments
    \\*
    \midrule
        1.10.0
            & Leomar Durán
            & 2022-03-25t23:57
            & design and implementation challenges, started anticipated project outcomes and impacts
    \\*
        1.9.0
            & Leomar Durán
            & 2022-03-25t23:46
            & data science fundamentals, project objective and constraints
    \\*
        1.8.0
            & Leomar Durán
            & 2022-03-25t22:31
            & finished overview of problem, flush level 1 lists
    \\*
        1.7.0
            & Leomar Durán
            & 2022-03-25t22:19
            & moved description of dataset to proposed work
    \\*
        1.6.0
            & Leomar Durán
            & 2022-03-25t22:16
            & added revision history
    \\*
        1.5.0
            & Leomar Durán
            & 2022-03-25t21:47
            & added more about the dataset, grammatical fixes
    \\*
        1.4.0
            & Leomar Durán
            & 2022-03-25t02:45
            & started bibliograph, problem statement
    \\*
        1.3.0
            & Leomar Durán
            & 2022-03-25t02:10
            & motivation, flushed heading numbers into margin
    \\*
        1.2.0
            & Leomar Durán
            & 2022-03-25t02:05
            & sections, title, students
    \\*
        1.1.0
            & Leomar Durán
            & 2022-03-25t01:55
            & page specifications
    \\*
        1.0.0
            & Leomar Durán
            & 2022-03-25t01:51
            & starting proposal
    \\*
    \bottomrule
\end{tabularx}

\section{Introduction section}

\subsection{Motivation}

The sciences of geomatics and land surveying interest me as hobbies.
I really enjoy the idea of collecting data about the terrain,
whether it be rural or urban,
and working with that data to find solutions to problems
or even just for fun.

\subsection{The Problem}

\subsubsection{Overview}

An image of a terrain is given to a computer which will make a decision on the fly based on the type of terrain.
We will train a model that will be used by the computer to identify this terrain.

This sort of decision might be involved in deciding if the terrain would be appropriate for developing a building thereon.
A preliminary sweep by a machine may save on costs of having an engineer waste time looking for land to develop.
Another example of making this decision may be helpful for automatic landing software that will be used to safely land aircraft on stable terrain.
A third example is that combined with time series data, we can predict different types of natural weather-related phenomena,
such as draughts, floods and earthquakes.

\subsubsection{Data science fundamentals}

This problem involves multi-class classification.
We will classify the terrain according to features such as whether the area is
urban, densely forested, mountainous, or contains water
for \(4\) disjoint classes.

We will evaluate the results using
accuracy, recall, precision and the F1 score.

\subsubsection{Project objective and constraints}

For this project, we hope to train a model to learn different \(4\) disjoint classes of terrain, and then classify a test sample.

The algorithm that we pick has to deal well with the curse of dimensionality,
as there will be
\(
    \SI[parse-numbers=false]{\brao{28\times28}\!}{\pixel\per\example}
    \times \SI4{\channel\per\pixel}
    = \SI{3136}{\channel\per\example}
\).

Ideally the solution would also perform well for multiple clases,
but this is less of an issue than dimensionality.

\subsection{Related works}

One of the historical approaches to this problem is that by \textcite{Basu2015a} themselves,
who
used a combination of computer vision and neural networks.

\textcite{IBM2020a} compares computer vision with human sight as well as artificial intelligence,
making the analogy that computer vision is to seeing as artificial intelligence is to thinking.

However, \textcite{IBM2020a} also clarifies the disadvantages of computer vision to traditional machine learning models.
Specifically, ``\tbrac{c}omputer vision needs lots of data. It runs analyses of data over and over until it discerns distinctions and ultimately recognize images.''
That is to say that computer vision has high time and spatial complexity.
Because of the amounts of data required,
it will require much storage,
and the same compounded by the number of iterations that must be performed,
training a computer vision model will require much more time.
IBM explains that the scale needed for time and storage is such that few organizations have the necessary infrastructure,
and as a result, many use a service such as IBM's to perform computer vision \cite{IBM2020a}.

\section{Proposed work}

We are given
The SAT-4 airborne dataset%
\cite{Basu2015a}.
This data is hosted by the Louisiana State University's Division of Computer Science and Engineering%
\footnote{%
    \tangle{\url{http://csc.lsu.edu/~saikat/deepsat/}}%
}
and can be downloaded directly from the Google Drive%
\footnote{%
    \tangle{\url{https://drive.google.com/u/0/uc?export=download&confirm=sWVM&id=0B0Fef71_vt3PUkZ4YVZ5WWNvZWs}}%
}
along with the SAT-6 airborne dataset.

The dataset consists of \(\num{400000}\) example tiles
taken from satellite imagery originally from the National Agriculture Imagery Program \tbrao{NAIP} dataset.
Each sample has features representing the pixels of a \(\SI[parse-numbers=false]{\brao{28\times28}\!}\pixel\) image 
multiplied by the channels for red, green, blue and near infrared \tbrao{NIR}.
According to \textcite{Basu2015a},
these tiles represent ``different landscapes like rural areas, urban areas, densely forested, mountainous terrain, small to large water bodies",
so these as a disjoint set of landscapes would make for appropriate labels.

Our proposed solution is a multi-class logistic regression.

\subsection{Design and implementation challenges}

A challenge to this solution is the size of the dataset.
Because of its size \tbrao{about \(\SI3{\giga\byte}\)}, we expect long processing times.
One possible solution to this challenge may be to reduce the datasize
from \(\num{400000}\) to a more managable number such as
\(\num{20000}\).

Another issue that we will run into is deciding the best way to split the classes for the multi-class classification.

\subsection{Anticipated project outcomes and impacts}

An anticipated outcome is a model that can identify the types of terrains accurately from the given dataset.

\section{Timeline}

\printbibliography

\end{document}
